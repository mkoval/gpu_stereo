\documentclass{article}
\usepackage{float}
\usepackage{fullpage}
\usepackage{graphicx}
\usepackage{subfig}

\title{
	\textbf{ECE452 Final Project} \\
	Parallelized Stereo Reconstruction
}
\author{
	Michael Koval \\
	mkoval@eden.rutgers.edu
}
\date{December 16, 2010}

\begin{document}
\maketitle

\section{Project Description}
Stereo vision is the process of extracting depth information from two images of
the same scene. Humans are able to unconsciously do this, but mimicking this
success using digital cameras is quite challenging and is frequently studied by
computer vision researchers. At its heart, stereo vision is the problem of
finding corresponding points in two images of the same scene. There are two
primary types of algorithms used for this purpose: region-based \textit{graph
cuts} matching algorithms and pixel-based \textit{block matching} algorithms.
This project focuses on the latter, specifically the \textit{sum of absolute
difference} (SAD) or \textit{sum of square difference} (SSD) algorithm that
was first implemented on SRI's Small Vision Module (Konolige, 1998).

This algorithm begins by computing the Laplacian of Gaussian (LoG) of the input
images by convolving both images with the appropriate kernel. For each pixel in
the filtered left image, Konolige's SAD-BM algorithm searches for the
corresponding pixel in the right image that minimizes the L1 norm between the
windows centered at the two pixels in the LoG-transformed images. The number of
pixels between the feature in the left and right images is called that pixel's
\textit{disparity} and is inversely proportional to distance; i.e. points with
high disparity are closer to the camera than those with low disparity.
Therefore, the image consisting of these disparity values---known as the
\textit{disparity map}---encodes the depth of each pixel in the scene.

Unfortunately, this algorithm is extremely computationally intensive:
processing an image of size $w \times h$ with maximum disparity $d$ has a
worst-case time complexity of $O(w h d)$, which is approximately $O(w^3)$ under
real conditions. Thankfully, this algorithm is also a great candidate for
parallelization: the same operations are repeated on each pixel in the image.
In this project, I will implement several versions of the SAD-BM algorithm and
compare their performance to the highly-optimized implementation included in
OpenCV, an open-source computer vision library.

\section{Objectives}
\begin{itemize}
\item Benchmark OpenCV's CPU implementations of SAD-BM
\item Write a custom serial CPU implementation of SAD-BM as a benchmark
\item Evaluate the performance increases gained from serial optimization
\item Parallelize the custom CPU implementation of SAD-BM using OpenMP
\item Benchmark OpenCV's GPU implementations of SAD-BM
\item Write a custom parallel GPU implementation of SAD-BM
\item Evaluate the performance of each algorithm under a variety of conditions
\end{itemize}

\section{Work Done}
\begin{itemize}
\item Benchmark OpenCV's CPU and GPU implementations of SAD-BM
\item Write a custom serial CPU implementation of SAD-BM as a benchmark
\item Benchmark OpenCV's GPU implementations of SAD-BM
\end{itemize}

\begin{figure}[H]
	\centering
	\subfloat[Left Input]{ \includegraphics[width=0.33\textwidth]{figures/input_left}}
	\subfloat[Right Input]{\includegraphics[width=0.33\textwidth]{figures/input_right}}
	\caption{
		Stereo pair used to evaluate the SSD-BM algorithm. Note that the two
		images are already rectified.
	}
\end{figure}

\begin{figure}[H]
	\centering
	\subfloat[OpenCV CPU]{\includegraphics[width=0.33\textwidth]{figures/opencv_cpu}}
	\subfloat[OpenCV GPU]{\includegraphics[width=0.33\textwidth]{figures/opencv_gpu}}
	\subfloat[Custom CPU]{\includegraphics[width=0.33\textwidth]{figures/custom_cpu}}
	\caption{
		Comparison between OpenCV and the custom implementations of SSD-BM. The differences
		between the OpenCV CPU implementation and the other two implementations are due to
		additional filtering and post-processing and are not related to the block-matching
		algorithm.
	}
\end{figure}

\begin{table}
	\centering
	\begin{tabular}{|c|c|}
	\hline
	Implementation & Time (s) \\
	\hline
	OpenCV CPU & 0.01418 \\
	OpenCV GPU & 0.06902 \\
	Custom CPU & 11.4897 \\
	\hline
	\end{tabular}
	\caption{
		Preliminary benchmarks of OpenCV and the custom CPU implementation. OpenCV
		dramatically beats the custom implementation because of it has been
		hand-optimized and makes extensive use of SSE2 for SIMD on small integer
		datatypes. This performance gap will close once the necessary optimizations
		have been made to the custom implementation.
	}
\end{table}

\section{Remaining Work}
\begin{itemize}
\item Evaluate the performance increases gained from serial optimization
\item Parallelize the custom CPU implementation of SAD-BM using OpenMP
\item Write a custom parallel GPU implementation of SAD-BM
\item Evaluate the performance of each algorithm under a variety of conditions
\end{itemize}

\section{Expected Results}
As the preliminary results show, OpenCV's CPU implementation is considerably
faster than the OpenCV GPU implementation and the custom implementation. This
is not surprising for two reasons: (1) the GPU implementation was evaluated on
a low-end notebook graphics card and (2) the custom CPU implementation has not
yet been optimized. After evaluating the GPU implementation on better hardware,
I expect the OpenCV GPU implementation to dramatically outperform the
corresponding CPU implementation. Similarly, I expect the custom SAD-BM
algorithm to be closer to the performance of the OpenCV CPU implementation
after the necessary optimizations have been made.

\end{document}
